\newif\iffull
\newif\ifnotes
\newif\iflabels
\newif\ifanonymous
\newif\ifspace
\newif\ifcameraready
\newif\ifbibtex

%%% Settings
%% Full version of the paper
\fulltrue
%% Show notes and todos
\notestrue
%% Print labels in the margins
% \labelstrue
%% Hide authors
% \anonymoustrue
%% Enable hacks to save space
% \spacetrue
%% Disable some tweaks that the publisher may not like
% \camerareadytrue
%% Use legacy BibTeX instead of Biber
% \bibtextrue

\documentclass[
  % Pass either "a4paper" or "letterpaper" explicitly.
  % (If the paper size is not specified, it will depend on the default
  % set in the (La)TeX installation.)
  a4paper,
  orivec,oribibl]{llncs}
\unless\ifcameraready
    \usepackage{llncsstandalone}  % local package
    % This is equivalent to 3cm margins (and the page shifted 0.5cm up)
    % on a4paper but gives an identical textblock on letterpaper.
    \usepackage[twoside=false, textwidth=150mm, textheight=237mm, vmarginratio=5:7, footskip=13mm]{geometry}
\fi
\usepackage[save]{silence}
\usepackage[T1]{fontenc}
\usepackage{lmodern}

%%% Space savings.
%%% The evil stuff. Uncomment manually.
\ifspace
  %% Scale line spacing. This becomes apparent at ~0.985 and below.
  % \renewcommand\baselinestretch{0.99}
  %% Always use inline style for \sum, \prod, ...
  % \PassOptionsToPackage{nosumlimits}{amsmath}
\fi

%%% Misc early packages
\usepackage[english]{babel}
\usepackage{csquotes}
\usepackage[babel,final]{microtype}
\usepackage{xpunctuate}
\usepackage{pifont}  % Dingbats
\usepackage{tcolorbox}
\tcbset{colback=white, size=title, boxsep=2mm}

%%% Colors
\usepackage[dvipsnames,svgnames,x11names]{xcolor}
\definecolor{bitcoin-orange}{RGB}{246, 145, 29}

%%% Math
\usepackage{amsmath}  % mostly for \qedhere
\usepackage{fixcmex}
% amsthm's proof environment conflicts with that of llncs
\let\lncsproof\proof
\let\lncsendproof\endproof
\let\proof\relax
\let\endproof\relax
\usepackage{amsthm}
\usepackage{mathtools}  % for \coloneq and \DeclarePairedDelimiter

%%% Notes and Todos
\NewDocumentCommand\newuser{m m m}{%
  \expandafter\NewDocumentCommand\csname #1note\endcsname{s O{} +m}{
    % Ensure that todos after paragraph headings are properly displayed
    \quitvmode%
    \texorpdfstring{%
      \todo[color=#3, inline, caption={}, ##2]{\textbf{#2:} ##3}%
    }{(Note by #2: ##3)}%
  }}
\ifnotes
  \usepackage{todonotes}
\else
  \usepackage[disable]{todonotes}
\fi

\iflabels
  \usepackage[notref,notcite,color]{showkeys}
\fi

%%% Bibliography
\ifbibtex
  \PassOptionsToPackage{backend=bibtex}{biblatex}
\fi
\PassOptionsToPackage{
  date=year,
  maxnames=3,
  minnames=3,
}{biblatex}
\ifcameraready
  \usepackage[
  style=biblatex-lncs/lncs,
  ]{biblatex}
  % Disable shorthands, see https://github.com/plk/biblatex/issues/1427
  \DeclareFieldInputHandler{shorthand}{\def\NewValue{}}
\else
  \usepackage[
  style=alphabetic,
  maxbibnames=15,
  minbibnames=15,
  ]{biblatex}
\fi

%%% Printing ePrint identifiers
\newcommand\iacreprintname{Cryptology ePrint Archive}
%% Shorter alternative:
% \newcommand\iacreprintname{IACR ePrint}
% TODO Consider just "\textsc{IACR}" (like for URL and DOI).
\DeclareFieldFormat{eprint:iacr}{%
  \iacreprintname\addcolon\space
  \ifhyperref
    {\href{https://eprint.iacr.org/#1}{%
       \nolinkurl{#1}%
       \iffieldundef{eprintclass}
         {}
         {\addspace\texttt{\mkbibbrackets{\thefield{eprintclass}}}}}}
    {\nolinkurl{#1}%
     \iffieldundef{eprintclass}
       {}
       {\addspace\texttt{\mkbibbrackets{\thefield{eprintclass}}}}}}

\DeclareSourcemap{
  \maps[datatype=bibtex]{
    \map{
      \step[fieldsource=howpublished, match=\regexp{\ACryptology\x20ePrint\x20Archive,\x20Report\x20(\d+)/(\d+)\z}, final]
      \step[fieldset=eprint, fieldvalue={$1/$2}]
      \step[fieldset=eprinttype, fieldvalue={iacr}]
      \step[fieldset=howpublished, null]
    }
    \map{
      \step[fieldsource=url, match=\regexp{https://eprint\.iacr\.org/(\d+)/(\d+)}, final]
      \step[fieldset=eprinttype, fieldvalue={iacr}]
      \step[fieldset=eprint, fieldvalue={$1/$2}]
      \step[fieldset=url, null]
    }
  }
}

\addbibresource{cryptobib/abbrev3.bib}
% "crypto.bib" will be treated specially: We use the biber-wrapper.sh
% script to extract the cited entries from crypto.bib into a temporary
% bib file and then call biber on it, to avoid that biber takes very
% long to process the huge crypto.bib. All of this should happen
% automatically and transparently if you use latexmk.
\addbibresource{cryptobib/crypto.bib}
\addbibresource{add.bib}

%%% Save space in references
\AtEveryBibitem{
  % See Section 2.2.2 in the BibLaTeX manual for the field types
  \unless\iffull
    \clearfield{doi}
  \fi
  \clearlist{location}
  \clearname{editor}
  \clearfield{pages}
  \ifentrytype{inproceedings}{
      \clearfield{year}
      \clearfield{volume}
      \clearlist{publisher}
      \clearfield{series}
    }{}
}

%%% hyperref
\unless\ifcameraready
  \PassOptionsToPackage{pdfusetitle}{hyperref}
\fi
\usepackage{hyperref}
\hypersetup{
  hypertexnames=false,
}
\unless\ifcameraready
  \hypersetup{
    colorlinks=true,
    allcolors=blue,
    bookmarksopen,
  }
\usepackage{bookmark}
\fi

%%% Cross-references
\usepackage[capitalize, nameinlink]{cleveref}
% Force "eq.", "line" and "item" to be lowercase.
% "eq.~" is a bit involved: We'll \crefformat to include the "~",
% but we don't want to abbreviate at the beginning of a sentence.
% All these definitions are based on "equation" examples in the
% cleveref manual.
\crefformat{equation}{#2eq.~(#1)#3}
\Crefformat{equation}{#2Equation~(#1)#3}
\crefmultiformat{equation}{eqs.~(#2#1#3)}{ and~(#2#1#3)}{, (#2#1#3)}{ and~(#2#1#3)}
\Crefmultiformat{Equation}{Equations~(#2#1#3)}{ and~(#2#1#3)}{, (#2#1#3)}{ and~(#2#1#3)}
\crefrangeformat{Equation}{eqs.~(#3#1#4) to~(#5#2#6)}
\crefrangeformat{Equation}{Equations~(#3#1#4) to~(#5#2#6)}
\Crefname{equation}{Equation}{Equations}  % Don't abbreviate at start of sentence
\crefname{line}{line}{lines}
\Crefname{line}{Line}{Lines}
\crefname{enumi}{item}{items}
\Crefname{enumi}{Item}{Items}

%%% Alternative for cross-references
% \usepackage{zref-clever}
% \zcsetup{cap}
% % Replacement for \namecref
% \newcommand{\zcnameref}[1]{\zcref*[noref,nocap]{#1}}
% \zcRefTypeSetup{assumption}{
%   Name-sg = Assumption ,
%   name-sg = assumption ,
%   Name-pl = Assumptions ,
%   name-pl = assumptions ,
% }
% \zcRefTypeSetup{equation}{
%   abbrev = true ,
% }

%%% Additional theorem-like environments
\newtheorem{assumption}{Assumption}
\Crefname{assumption}{Assumption}{Assumptions}
\newtheorem{construction}{Construction}
\Crefname{construction}{Construction}{Constructions}

%%% Cryptocode
%%% TODO Extract this into a "mycryptocode.sty" package
\WarningFilter*[theH]{latex}{Command `\string\the}%
\ActivateWarningFilters[theH]
\usepackage[
  lambda,
  advantage,
  operators,
  adversary,
  landau,
  probability,
  sets,
  % notions,
  % logic,
  % ff,
  % mm,
  % primitives,
  % oracles,
  events,
  % complexity,
  asymptotics,
  keys,
]{cryptocode}
\DeactivateWarningFilters[theH]

% Better version of \pcfixcleveref
% (see https://github.com/arnomi/cryptocode/pull/15)
\IfPackageLoadedTF{cleveref}{
  \AtBeginDocument{
    \pcfixhyperref%
    \makeatletter
    \crefalias{@pclinenumber}{line}%
    \makeatother
    % TODO Consider further improvements for the interaction between
    % crytocode and cleveref, e.g., referencing games, see: https://tex.stackexchange.com/questions/623163/cryptocode-cref-should-link-to-a-game-and-write-its-name/623196#623196
  }
}{}

% Fix \gamechange to expect math mode contents
% (see https://github.com/arnomi/cryptocode/pull/21 )
\renewcommand{\gamechange}[2][gamechangecolor]{%
  {\setlength{\fboxsep}{0pt}\ifmmode%
      \mathchoice{%
        \colorbox{#1}{$\displaystyle#2$}%
      }{%
        \colorbox{#1}{$#2$}%
      }{%
        \colorbox{#1}{$\scriptstyle#2$}%
      }{%
        \colorbox{#1}{$\scriptscriptstyle#2$}%
      }%
    \else{#2}\fi}%
}

% Improved paired delimiters
% Usage: \abs{x} for auto-scaling, or \abs[\bigg] for manual scaling
\makeatletter
\RenewDocumentCommand\floor{om}{\IfValueTF{#1}{\pc@floor[#1]{#2}}{\pc@floor*{#2}}}
\RenewDocumentCommand\ceil{om}{\IfValueTF{#1}{\pc@ceil[#1]{#2}}{\pc@ceil*{#2}}}
\RenewDocumentCommand\Angle{om}{\IfValueTF{#1}{\pc@Angle[#1]{#2}}{\pc@Angle*{#2}}}
\RenewDocumentCommand\abs{om}{\IfValueTF{#1}{\pc@abs[#1]{#2}}{\pc@abs*{#2}}}
\RenewDocumentCommand\norm{om}{\IfValueTF{#1}{\pc@norm[#1]{#2}}{\pc@norm*{#2}}}
\DeclarePairedDelimiter{\pc@bracks}{\lbrack}{\rbrack}
\NewDocumentCommand\bracks{om}{\IfValueTF{#1}{\pc@bracks[#1]{#2}}{\pc@bracks*{#2}}}
\NewDocumentCommand\pr{om}{\Pr\bracks[#1]{#2}}
\makeatother

%%% Append "." to subsubsection and paragraph headings
\let\llncssubsubsection\subsubsection
\renewcommand{\subsubsection}[1]{\llncssubsubsection{#1.}}
\let\llncsparagraph\paragraph
\renewcommand{\paragraph}[1]{\llncsparagraph{#1.}}


%%% We typically use \NewDocumentCommand instead of \newcommand(*) in
%%% order to obtain robust commands, which is usual in modern LaTeX.
%%% In many cases below, it would be okay to use \newcommand, e.g., as
%%% the underlying commands are robust but we mostly stick to
%%% \NewDocumentCommand for consistency.
%%%
%%% See https://latex3.github.io/help/documentation/usrguide.pdf for
%%% background.

%%% We use := as assignment operator
\NewDocumentCommand\defeq{}{\coloneq}
\NewCommandCopy\origgets\gets
\RenewCommandCopy\gets\defeq

%%% cryptocode tweaks
% Multi-letter variables, e.g., "ctr"
\NewDocumentCommand\var{m}{\ensuremath{\mathit{#1}}}
% Make keys and polynomials look like other identifiers.
% (No idea why cryptocode treats them specially.)
\RenewDocumentCommand\pckeystyle{m}{\ensuremath{\var{#1}}}
\RenewDocumentCommand\pcpolynomialstyle{m}{\ensuremath{\var{#1}}}
% Add 1pt padding to \gamechange (based on original definition).
\renewcommand{\gamechange}[2][gamechangecolor]{%
  {\setlength{\fboxsep}{1pt}%
    \colorbox{#1}{#2}}}
% Shorthand for \pcalgostyle
\NewDocumentCommand\algo{}{\pcalgostyle}
% Oracles
\NewDocumentCommand{\mathsc}{m}{{\normalfont\textsc{#1}}}
\RenewDocumentCommand\pcoraclestyle{m}{\ensuremath{\mathsc{#1}}}
\NewDocumentCommand\oracle{m}{\pcoraclestyle{#1}}
% Games
\newcommandx{\game}[4][3=\adv,4=(\secpar)]{{\operatorname{#1}_{#2}^{#3}#4}}
\newcommand{\Game}{\algo{Game}}
% Misc
\newcommand{\pcsc}{\,;~}

%%% Algorithms and named schemes
%%%
%%% Uncomment these manually as needed.
% \NewDocumentCommand\grgen{}{\algo{GrGen}}
% \NewDocumentCommand\gparam{}{(\GG,p,g)}

% \NewDocumentCommand{\inpgen}{\algo{InpGen}}
% \NewDocumentCommand{\param}{\var{par}} % public parameters of the multisig scheme

% \NewDocumentCommand\setup{\algo{Setup}}
% \NewDocumentCommand\keygen{\algo{KeyGen}}
% \NewDocumentCommand\sign{\algo{Sign}}
% \NewDocumentCommand\verify{\algo{Verify}}

% \NewDocumentCommand\musigtwo{\algo{MuSig2}}
% \NewDocumentCommand\musigtwoias{\algo{MuSig2\textnormal{-}IAS}}

%% Game Oracles
% \NewDocumentCommand{\dlo}{}{\oracle{DL}}

%% Random oracles
% \NewDocumentCommand{\hash}{\algo{H}}
% \NewDocumentCommand{\Hsig}{\algo{H}_{\mathrm{sig}}}

%%% For easier typing
\NewDocumentCommand{\eg}{}{e.g\xperiod}
\NewDocumentCommand{\ie}{}{i.e\xperiod}
% Caution: \wlog is a TeX builtin that writes to the log.
\NewDocumentCommand{\wlg}{}{w.l.o.g\xperiod}

%%% Dingbats for checkmark and cross, e.g., for feature columns in tables
\NewDocumentCommand\cmark{}{\ding{51}}
\NewDocumentCommand\xmark{}{\ding{55}}

%%% Bitcoin symbol
% https://tex.stackexchange.com/a/112165, slightly improved
\NewDocumentCommand\btcsym{}{%
  \leavevmode
  \vtop{\offinterlineskip\upshape%\bfseries
    \setbox0=\hbox{B}%
    \setbox2=\hbox to\wd0{\hfil\hskip-.03em
    \vrule height .3ex width .15ex\hskip .08em
    \vrule height .3ex width .15ex\hfil}
    \vbox{\copy2\vskip-.01ex\box0}\vskip-.01ex\box2}}



%%% Title
%
% Springer wants:
% All words in titles should be capitalized except for conjunctions, prepositions
% (e.g. on, of, by, and, or, but, from, with, without, under), and definite/indefinite
% articles (the, a, an), unless they appear at the beginning. Formula letters are
% typeset as in the text.
%
% Hint: https://titlecaseconverter.com/
\title{\texorpdfstring{%
    My Title with Manual Line Breaks\\ and Some Special Chars like $\Lambda$%
  }{% Title for PDF metadata and bookmarks:
    My Title with Manual Line Breaks and Some Special Chars like Λ%
  }
}

%%% Authors
\unless\ifanonymous
  \unless\ifcameraready
    \RenewDocumentCommand\email{m}{\href{mailto:#1}{\texttt{\textcolor{black}{#1}}}}
    \RenewDocumentCommand\orcidID{m}{}
  \fi
  \author{\texorpdfstring{%
      Alice Author\inst{1}\orcidID{0000-0000-0000-0000}
      \and Bob Author\inst{2}\orcidID{0000-0000-0000-0000}
    }{% Authors for PDF metadata and bookmarks:
      Alice Author;
      Bob Author
    }}
  \institute{%
    Some Institute
    \and Some University
  }
\else
  \ifspace
    \author{\vspace{-4ex}}\institute{}
  \fi
\fi

%%% Notes and Todos
% Usage: \newuser{a}{Alice}{orange!20}
%
% The "!20" suffix creates a mix of 20% of the preceding color and 80%
% white. See https://latexcolor.com/ for predefined colors.
\newuser{a}{Alice}{orange!20}
\newuser{b}{Bob}{olive!20}

\begin{document}

\maketitle
\iffull
  \begin{center}
    \today
  \end{center}
\fi

%%% Abstract
\begin{abstract}
  \input{abstract.tex}
\end{abstract}

%%% Keywords
\ifcameraready
  \keywords{a keyword \and another one \and keywords are separated with \texttt{\textbackslash{}and}}
\fi

%%% Table of Contents
\iffull
  \tableofcontents
  \clearpage
\fi

%%% Uncomment this for usage notes on this LaTeX template
\setcounter{section}{-1}
\section{Template Usage Notes}
This is a brief overview.
See the respective package documentation for more details. (Hint: Try \texttt{texdoc -l <pkg>} if you use TeXLive or check out \url{https://texdoc.org/}.)
The source code of this file (\texttt{\CurrentFile}) may be instructive, too.



\subsection{Compilation}
It's recommended to use \texttt{latexmk}.
A configuration file is included, so simply running \texttt{latexmk} from the \texttt{src} directory without arguments is enough.
Use \texttt{latexmk -C} to clean auxiliary files, which are all stored in the \texttt{latex.out} directory.
The root source file is \texttt{main.tex}.


\subsection{Toggles}
A few toggles are provided at the top of \texttt{main.tex}:
\begin{quote}
  \begin{description}
    \item [\texttt{\textbackslash{}fulltrue}:]
          Full version of the paper. (Authors can also use \verb|\iffull| where appropriate.)
    \item [\texttt{\textbackslash{}notestrue}:]
          Enable notes and todos
    \item [\texttt{\textbackslash{}labelstrue}:]
          Print labels
    \item [\texttt{\textbackslash{}anonymoustrue}:]
          Hide authors
    \item [\texttt{\textbackslash{}spacetrue}:]
          Enable hacks to save space
    \item [\texttt{\textbackslash{}camerareadytrue}:]
          Disable some tweaks that the publisher may not like
  \end{description}
\end{quote}


\subsection{Theorem-like Environments}
The \texttt{llncs} class defines the environments
\texttt{corollary}, \texttt{definition}, \texttt{lemma}, \texttt{proposition}, and \texttt{theorem},
as well as \texttt{proof} and \texttt{claim} with a different formatting.
This template adds \texttt{assumption}.

\begin{theorem}[My Theorem]\label{thm:my-thm}
  \begin{equation}\label{eq:my-thm}
    C = g^mh^r.
  \end{equation}
  \begin{equation}\label{eq:my-thm2}
    \Pr[\bad] \le \negl.
  \end{equation}
  \begin{equation}\label{eq:my-thm3}
    a = b
  \end{equation}
\end{theorem}
\begin{proof}[Proof (handwavy)]
  Believe, me, \cref{eq:my-thm} holds.
  Here's an equation to demonstrate that \verb|\qedhere| works:
  \[
    c = m+rx . \qedhere
  \]
\end{proof}

\begin{assumption}[VERY-HARD-PROB]\label{ass:very-hard}
  Getting $10\,\btcsym$ is a very hard work.
  See also \cref{thm:my-thm}.
\end{assumption}

\begin{remark}\label{rem:add-env}
  There are additional environments
  \texttt{case}, \texttt{conjecture}, \texttt{example}, \texttt{exercise}, \texttt{note}, \texttt{problem}, \texttt{property}, \texttt{question}, \texttt{remark}, and \texttt{solution}.
  These have a different formatting (like this \lcnamecref{rem:add-env}).
\end{remark}


\subsection{Citations and Bibliography (\texttt{biblatex})}
See \texttt{README.md} for instructions how to install \href{https://cryptobib.di.ens.fr/}{\texttt{crypto.bib}}.

This template defaults to BibLaTeX's modern Biber backend, which can handle UTF-8 and which will enable all BibLaTeX features.
But since Biber tends to be slow and will need over a minute to parse the huge \texttt{crypto.bib},\footnote{See \url{https://github.com/plk/biber/issues/371} for background.}
we use a wrapper around Biber that automatically extracts only the cited entries from \texttt{crypto.bib} to a temporary BibTeX file that will then be processed by Biber.
All of this should happen automatically and transparently if you use \texttt{latexmk}.

You can use \verb|\textcite| to typeset author names automatically, \eg, \textcite{CCS:BelNev06,EPRINT:NicRufSeu20}.

It's a good idea to define the \texttt{shorthand} field in the bib file for BIPs, RFCs and similar documents, \eg, \verb|shorthand = {BIP340}|. By the way, have you heard about xonly keys~\cite{add:bip-schnorr}?


\subsection{References (\texttt{cleveref})}
\begin{enumerate}
  \item\label{item:one}
        Use \verb|\cref{label,maybe-another-label}| to insert a reference, \eg, \cref{sec:intro,thm:my-thm,ass:very-hard}.
        The prefix, \eg, ``Theorem'' or ``Appendix'' will be added automatically.
  \item
        References to list items, and line numbers (\cref{line:false,item:one}) are not capitalized.
        The word ``equation'' (or ``eq.'') is omitted in front of a reference to an equation, \eg, \cref{eq:my-thm,eq:my-thm2}.
        Use \verb|\Cref| at the beginning of a sentence to force upper-case and to force typesetting the word ``Equation''.
        For example:
        \Cref{line:false,item:one}. But: \Cref{item:one}. And: \Cref{eq:my-thm,eq:my-thm2,item:one}.
        Ranges are typeset with an en dash.
        \emph{By the way, equation numbers are always typeset upright,
        \eg, \cref{eq:my-thm,eq:my-thm2,eq:my-thm3},
        even when the surrounding text is italic.}
\end{enumerate}


\subsection{Notes and Todos (\texttt{todonotes})}
\newuser{readme}{Template}{orange!20}
\readmenote{
  Set \texttt{\textbackslash{}notestrue} to enable displaying notes.
  You can define a user with \texttt{\textbackslash{}newuser}, \eg,  \texttt{\textbackslash{}newuser\textbraceleft{}s\textbraceright{}\textbraceleft{}Satoshi\textbraceright{}\textbraceleft{}orange\textbraceright{}}, see the source code.
  This will create the user-specific command \texttt{\textbackslash{}snote} for inserting notes.

  A note can span multiple paragraphs.
}


\subsection{Macros \texorpdfstring{(\eg, $\btcsym$, $\algo{Sign}$, $\pk$, $\defeq$, and $\xmark$) }{}Are Mostly Robust}
Thus, they can be used in moving arguments such as section headings.
If not, try to put a \verb|\protect| in front of them.

A drawback of robustness is that PDF metadata such as bookmarks created from section headings,
will contain the \LaTeX{} source code instead of the expanded macro.
You probably want to use \verb|\texorpdfstring{<actual LaTeX code>}{<plaintext for PDF bookmark>}| to fix that.


\subsection{Pseudocode (\texttt{cryptocode})}
See the source code of \cref{fig:scheme}.
\begin{figure}[bthp]
  \begin{center}
    % If you omit the width, then it will be 100%.
    % Try to specify boxsep=1mm or smaller if space is tight
    \begin{tcolorbox}[width=10cm]
      \begin{pchstack}[center]
        \begin{pcvstack}
          \procedure[linenumbering]{$\algo{Alg}(\secpar)$}{%
            \pcbox{x \sample \ZZ_p \pcsc \pcreturn \sk + x} \\
            \pcreturn \pcfalse \label{line:false}
          }
          \pcvspace
          \procedure[headlinesep=1pt]{$\game{\Game~\algo{Coll}}{\algo{H}}$}{%
            \mathellipsis
          }
        \end{pcvstack}
        \pchspace[1em]
        \begin{pcvstack}
          \procedure{$\algo{Foo}(\secpar)$}{%
            x \gets 5\\
            \gamechange{\pcassert \pcfalse}
          }
          \pcvspace
          \procedure[headlinesep=1pt]{$\oracle{Bar}(\sk, \pk)$}{%
            y \sample \ZZ_p
          }
        \end{pcvstack}
      \end{pchstack}
    \end{tcolorbox}
  \end{center}
  \caption{My great scheme\label{fig:scheme}}  % \label goes inside \caption
\end{figure}

\clearpage


\input{intro.tex}
\section{Preliminaries}\label{sec:prelim}


%%% Appendix
\appendix
\crefalias{section}{appendix}

\section{An Appendix}\label{sec:test-appendix}
Many people don't follow this rule, but consider putting the appendix before the references.
The idea is that the references will be easy to find: They're always at the end.

%%% Acknowledgements
% \section*{Acknowledgements}
% We thank...

%%% Changelog
\iffull
  \phantomsection
  \addcontentsline{toc}{section}{Changelog}

  \section*{Changelog}\label{sec:changelog}
  \begin{description}
    \item[2022-03-30]\
          \begin{itemize}
            \item First public draft.
          \end{itemize}
  \end{description}
  % TODO Why is spacing wrong without this?
  \vspace{-2em}
\fi

%%% Bibliography
\phantomsection
\addcontentsline{toc}{section}{References}
\printbibliography\label{sec:bib}

\end{document}
